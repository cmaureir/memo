\frame{
	\frametitle{Procesos de desarrollo de Software de carácter general}
	\begin{itemize}
	    \item Importancia de contar con un proceso de desarrollo a la hora de crear software.
	    \item Criterios \textit{universalmente} aceptados:
	    \begin{itemize}
	      \item Modelo Cascada
		  %\begin{itemize}
		  % \item No permite depuración.
		  % \item Se puede arreglar diseño de software o código sólo al final.
		  %\end{itemize}
	      \item Modelo Iterativo - Incremental 
		  % \begin{itemize}
		  % \item Permite depuración en ciertas etapas del proceso.
		  % \item Es posible detectar y corregir errores de diseño y/o código durante el proceso.
		  %\end{itemize}
	    \end{itemize}
	    \item Criterios conocidos como \textit{tradicionales}.
	    \item Basados en otras ingenierías
	     \begin{itemize}
	      \item Planificación total antes de comenzar el ciclo de desarrollo.
	      \item Ideales para entornos/situaciones de poca variación (Sistemas operativos, Aplicaciones de Escritorio).
	     \end{itemize}

	     
	     

	\end{itemize}
}