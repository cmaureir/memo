%En este apartado se verá la información recapitulada de la herramienta web Jquery,

Desde el portal [1], se pueden acceder a páginas relacionadas con Jquery.
Destaca el core de jquery (jquery.com) y el ui (jqueryui.com)\\

jQuery es una biblioteca o framework de Javascript, creada inicialmente por John Resig, que permite simplificar 
la manera de interactuar con los documentos HTML, manipular el arbol DOM, manejar eventos, desarrollar animaciones 
y agregar interacción con la tecnología AJAX a páginas web. Fue presentada el 14 de enero de 2006 en el BarCamp NYC.[4]\\

Cuando un desarrollador tiene que utilizar Javascript, 
generalmente tiene que preocuparse por hacer scripts compatibles con varios navegadores y para ello tiene que 
incorporar mucho código que lo único que hace es detectar el browser del usuario, para hacer una u otra cosa 
dependiendo de si es Internet Explorer, Firefox, Opera, etc. jQuery es donde más nos puede ayudar, puesto que 
implementa una serie de clases (de programación orientada a objetos) que nos permiten programar sin preocuparnos 
del navegador con el que nos está visitando el usuario, ya que funcionan de exacta forma en todas las plataformas 
más habituales. [2]\\

Además, todas estas ventajas que sin duda son muy de agradecer, con jQuery se obtenienen de manera gratuita, ya que 
el framework tiene licencia para uso en cualquier tipo de plataforma, personal o comercial. Para ello simplemente 
se debe incluir las páginas un script Javascript que contiene el código de jQuery, que se puede descargar 
de la propia página web del producto y comenzar a utilizar el framework.\\

El archivo del framework ocupa poco espacio (algunos KB solamente), lo que es bastante razonable y no retrasará mucho 
la carga de la página. Además, el servidor lo enviará al cliente la primera vez que visite una página del sitio. 
(Se envía el archivo jquery de forma directa al cargar la página con jquery la primera vez.)
En siguientes páginas el cliente ya tendrá el archivo del framework, por lo que no necesitará transferirlo y lo tomará desde 
la caché. Con lo que la carga de la página sólo se verá afectada por el peso de este framework una vez por usuario.\\

%Las ventajas a la hora de desarrollo de las aplicaciones, así como las puertas que nos abre jQuery compensan extraordinariamente 
%$el peso del paquete. 

Cabe destacar que jquery no está solo en el mercado, es decir cuenta con ferrea competencia. Sin embargo es bastante popular.
Además, es un producto serio, estable, bien documentado y con un gran equipo de desarrolladores a cargo de la mejora y 
actualización del framework. Otra cosa muy interesante es la dilatada comunidad de creadores de plugins o componentes, 
lo que hace fácil encontrar soluciones ya creadas en jQuery para implementar asuntos como interfaces de usuario, galerías, 
votaciones, efectos diversos. \\


jQuery es una herramienta imprescindible para desarrollar  sin tener que 
complicarse con los niveles más bajos del desarrollo, ya que muchas funcionalidades ya están implementadas, o bien las 
librerías del framework permitirán realizar la programación mucho más rápida y libre de errores.\\

Lo más complicado de jQuery es aprender a usarlo, igual que pasa con cualquier otro framework Javascript. Requerirá del desarrollador 
habilidades avanzadas de programación, así como el conocimiento, al menos básico, de la programación orientada a objetos. Una 
vez aprendido las ventajas de utilizarlo compensarán más que de sobra el esfuerzo.\\

Es posible acceder a un completo manual en [3].\\

Existe una versión para dispositivos móbiles según [5]. Con los dispositivos móviles se han multiplicado el número de 
navegadores y de plataformas. Existen muchos fabricantes, de 
tablets y smartphones y diversos dispositivos con características distintas, como tamaños de pantallas, sistemas operativos
diferentes y diversos navegadores basados en cada uno de ellos. Si antes con los navegadores para PCs había 
problemas de compatibilidad, considerando sólo los 3 sistemas operativos y 3 navegadores populares, ahora con los móviles 
todo se complica más.\\

%Porque el desarrollo de sitios web con jQuery Mobile es todavía más automático de lo que era trabajar con jQuery. Con 
%mucho menos código haces muchas más cosas.

Con los dispositivos móviles todavía se hace más necesario usar un framework, que  
facilita desarrollar una vez y ejecutar de manera compatible en todos los dispositivos. Además de ello, con 
jQuery Mobile se reducirá drásticamente el tiempo de desarrollo de un sitio web para dispositivos móbiles.\\

Se puede entender a jQuery Mobile como un plugin para jQuery puesto que es un producto que está basado en el propio 
framework Javascript jQuery. Por tanto, aquellas personas que ya conocen jQuery no van a tener que aprender nada nuevo, 
sino aplicar sus conocimientos y desarrollar fácilmente aplicaciones para móviles. Esto significa una gran ventaja, 
puesto que hay muchas personas que conocen jQuery y que van a poder pasarse sin prácticamente ningún problema al 
desarrollo para dispositivos.\\


\subsection{Plataformas}
Las plataformas soportadas por jquery, corresponden a los navegadores que sean compatibles con javascript. Cabe destacar
que  se ha realizado soporte para HTML5 incluso para IE 6/7 y 8.
%Además en su versión para dispositivos móbiles, se amplía
%\begin{itemize}
% \item listar los dispositivos
%\end{itemize}



[1] http://jquery.org/ (hora de consulta 29-11-12 13:15)

[2] http://www.desarrolloweb.com/manuales/manual-jquery.html (hora de consulta 29-11-12 14:43)

[3] http://www.desarrolloweb.com/articulos/introduccion-jquery.html (hora de consulta 29-11-12 14:04)

[4] http://www.todoblogger.com/2010/08/que-es-jquery.html (hora de consulta 29-11-12 14:47)

[5] http://www.desarrolloweb.com/articulos/introduccion-jquery-mobile.html (hora de consulta 29-11-12 14:51)


\subsection{Hello World}

Dentro del html, se invoca el archivo que contien a jquery. En la web hay muchos ejemplos de como utilizarlo.

\begin{verbatim}
    <script type="text/javascript" src="jquery-1.3.1.js"></script>
\end{verbatim}

En [6] hay un ejemplo de hola mundo utilizando Jquery
\begin{verbatim}
$(document).ready(function(){
$("a").click(function() {
alert("Hola Mundo con jQuery!");
});
});
\end{verbatim}


[6] http://lineadecodigo.com/jquery/hola-mundo-con-jquery/  (hora de consulta 29-11-12 15:29)
