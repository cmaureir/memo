\begin{enumerate}
 %\item Dadas las características de las herramientas, una posible idea es realizar un proyecto utilizando nodejs y jquery (o kendo en  su version gratis)
 \item Considerando que varias de las herramientas tienen versiones móviles, necesario determinar las aplicaciones webs orientadas 
 estos dispositivos se incluyen en esta memoria, o sólo se limitará a describirlas. Por ejemplo en qooxdoo se pueden hacer incluso
 aplicaciones de escritorio; v8 se puede emebeber en aplicaciones de c++-
 \item Al comenzar la investigación, asociaba v8 al servidor, debido a nodejs. Sin embargo, esto no es así, ya que v8 es un motor
 de javascript, lo cual no implica que esté sólo en el lado del servidor (si lo es nodejs). De hecho es utilizado por el navegador
 google chrome. Por lo mismo en las próximas entregas será necesario cambiar el orden de las tecnologías si se desea mantener 
 el criterio de \textit{backend-frontend}.
 \item Dentro del prototipo, creo que no habrá problemas al integrar jquery con kendo (este último basado en jquery). El uso
 de v8 y nodejs tampoco sería un problema (en teoría al menos) pues comparten el mismo vínculo que jquey y kendo (en parte).
 \item Queda pendiente el tema del manejo de conexiones concurrentes en nodejs. Sería interesante probar con alguna herramienta de 
 denegación de servicios, cuantas conexiones soporta.
 \item Surge la idea de usar Kendo para hacer la presentanción de defensa en HTML5 y los gráficos usando el ``data viz''
 de kendo.
 \item Respecto a JQuery, es necesario determinar el grado de profundidad de la investigación, quizas determinar que plugins 
 se utilizarán si o si (de todas formas, las interrogantes de este estilo se irán disipando cuando se establezca el caso de estudio).
\end{enumerate}


