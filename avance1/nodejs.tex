%Este apartado corresponde a la información recapitulada de la herramienta node.js:

Consiste en un framework de Javascript basado en el motor V8.\\

Que problema resuleve node.js

La meta número uno declarada de Node es "proporcionar una manera fácil para construir programas de 
red escalables". ¿Cuál es el problema con los programas de servidor actuales? Hagamos cuentas. En lenguajes 
como Java y PHP, cada conexión genera un nuevo hilo que potencialmente viene acompañado de 2 MB de memoria. 
En un sistema que tiene 8 GB de RAM, esto da un número máximo teórico de conexiones concurrentes de cerca de 
4.000 usuarios. A medida que crece su base de clientes, si usted desea que su aplicación soporte más usuarios,
necesitará agregar más y más servidores. Desde luego, esto suma en cuanto a los costos de servidor del negocio, 
a los costos de tráfico, los costos laborales, y más. Además de estos costos están los costos por los problemas 
técnicos potenciales — un usuario puede estar usando diferentes servidores para cada solicitud, así que cualquier 
recurso compartido debe almacenarse en todos los servidores. Por todas estas razones, el cuello de botella en 
toda la arquitectura de aplicación Web (incluyendo el rendimiento del tráfico, la velocidad de procesador y la 
velocidad de memoria) era el número máximo de conexiones concurrentes que podía manejar un servidor.\\

Node resuelve este problema cambiando la forma en que se realiza una conexión con el servidor. En lugar de 
generar un nuevo hilo de OS para cada conexión y de asignarle la memoria necesaria, cada conexión dispara una 
ejecución de evento dentro del proceso del motor de Node. Node también afirma que nunca se quedará en punto muerto, 
porque no se permiten bloqueos y porque no se bloquea directamente para llamados E/S. Node afirma que un servidor que 
lo ejecute puede soportar decenas de miles de conexiones concurrentes.\\

(NOTA, buscar sobre la programacion orientada a eventos)\\

Utiliza la máquina virtual de javascript de google conocida como v8 para interpretar el código javascript. 
La segunda característica más notable es que puede levantar servidores web en casi cualquier puerto que se le indique 
y generar llamadas tipo comet, osea que son conexiones persistentes que se comportan como sockets y que hacen posible 
la comunicación única bidirideccional entre el cliente y el servidor.[2]\\

Mas informacion y ejemplos en [3] [4]\\



\textbf{Para que sirve?}\\
Sirve para  manejar sitaciones dond hay alto tráfico de datos, y donde la cantidad de datos a procesar en el servidor, no 
sea demasiado alta (antes de enviar una respuesta al cliente). Algunos ejemplos son:

\begin{itemize}
 \item API RESTful: Una REST API es una API, o librería de funciones, a la que se accede por el protocolo HTTP. 
Una REST API, por tanto, se accede a través de direcciones web o URLs en las que enviamos los datos de nuestra consulta. 
Como respuesta a la consulta sobre el REST API se obtienen datos en diferentes formatos, como pueden ser texto plano, 
XML, JSON, etc.[1]
 \item Estadisticas de videojuegos
\end{itemize}

%1) API RESTful: Una REST API es una API, o librería de funciones, a la que se accede por el protocolo HTTP. 
%Una REST API, por tanto, se accede a través de direcciones web o URLs en las que enviamos los datos de nuestra consulta. 
%Como respuesta a la consulta sobre el REST API se obtienen datos en diferentes formatos, como pueden ser texto plano, 
%XML, JSON, etc.[1]
%2) Fila de Twitter
%3) Estadistica de videojuegos



%Sobre el uso de módulos\\

Así como las personas se han acostumbrado al trabajar con Apache, es posible expandir la funcionalidad de Node 
instalando módulos. No obstante, los módulos que se pueden utilizar con Node mejoran en gran medida el producto, tanto, 
que es poco probable que haya alguien que utilice Node sin instalar por lo menos algunos módulos. Así de importantes se 
han tornado los módulos, hasta el punto de convertirse en parte esencial del producto completo [5].


NodeJS y NPM


\subsection{Plataformas}

En la página oficial [6], están habilitadas las descargas para:
\begin{itemize}
 \item Windows
 \item Mac OS X
 \item Linux
 \item SunOS
\end{itemize}

%nota: buscar acerca del bloqueo en el manejo en node.js


[1] http://www.desarrolloweb.com/wiki/rest-api.html (hora de cosulta 28-11-12 17:11)

[2] http://www.quantium.com.mx/2012/08/06/nodejs/ (hora de consulta 28-11-12 18:18)

[3] http://www.rmunoz.net/introduccion-a-node-js.html (hora de consulta 28-11-12 18:25)

[4] http://www.quizzpot.com/2011/01/como-instalar-node-js-y-escribir-primeros-programas/ (hora de consulta 28-11-12 18:01)

[5] https://npmjs.org (hora de consulta 28-11-12 18:10)

[6] http://nodejs.org/download/ (hora de consulta 29-11-12 19:10) 


\subsection{Hello World}

Es necesario entrara a esta página http://nodejs.org/, install, y luego seguir los pasos
que salen en el archivo README.md
\begin{verbatim}
  ./configure
  make
  make install (pide permisos de root)
\end{verbatim}

y para el error del gcc 
\begin{verbatim}
 make
make -C out BUILDTYPE=Release V=1
make[1]: Entering directory `/home/joao/memoria-jfuentes-heramientas/nodejs/node-v0.8.14/out'
  g++ '-D_LARGEFILE_SOURCE' '-D_FILE_OFFSET_BITS=64' '-DENABLE_DEBUGGER_SUPPORT' 
  '-DV8_TARGET_ARCH_X64' -I../deps/v8/src  -Wall -pthread -m64 -fno-strict-aliasing -O2 
  -fno-strict-aliasing -fno-tree-vrp -fno-rtti -fno-exceptions -MMD -MF 
  /home/joao/memoria-jfuentes-heramientas/nodejs/node-v0.8.14/out/Release/.deps/
  /home/joao/memoria-jfuentes-heramientas/nodejs/node-v0.8.14/out/Release/obj.target/v8_base/deps/v8/src/accessors.o.d.raw  
  -c -o /home/joao/memoria-jfuentes-heramientas/nodejs/node-v0.8.14/out/Release/obj.target/v8_base/deps/v8/src/accessors.o 
  ../deps/v8/src/accessors.cc
make[1]: g++: Command not found
make[1]: *** [/home/joao/memoria-jfuentes-heramientas/nodejs/node-v0.8.14/out/Release/obj.target/v8_base/deps/v8/src/accessors.o] 
  Error 127
make[1]: Leaving directory `/home/joao/memoria-jfuentes-heramientas/nodejs/node-v0.8.14/out'
make: *** [node] Error 2

	yum install gcc-c++
\end{verbatim}

y seguir las instrucciones de la pagina oficial.\\

Una de las características de nodejs es el paso de parametros por url. Hay un ejemplo que
permite generar un número aleatorio desde 0 a $X$, siendo este un valor que se cambia en la url.\\

Cabe destacar que los modulos se importan a través de lineas como:
\begin{verbatim}
 var http = require("http");
 var url = require("url");
\end{verbatim}
las cuales son similares al ``include`` de otros lenguajes.

