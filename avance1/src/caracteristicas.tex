La idea de este documento es poder dejar en claro, las características, plataformas, relaciones de las diversas herramientas 
que se estudiarán en esta memoria de pregrado. Cabe destacar que este documento, por si solo, no sirve para efectos de 
presentación para la defensa de esta memoria, pues corresponde a un mero avance: recopilación de información, análisis y algunas 
conclusiones.\\

Se quiere dejar en claro los pasos necesarios para obtener, instalar y dar los primeros pasos de la diversas herramientas 
que se estudiarán.\\

Estas pruebas se realizarán en un sistema operativo GNU/Linux (específicamente Fedora 17 (x64)).\\


%Para ver como instalar cada una de ellas, por favor, vea el documento "instalaciones"

El orden en que se verán estas tecnologías no es trivial, pues van desde herramientas asociadas
al ``backend`` a las que están asociadas al ''frontend``.\\

Por lo tanto el orden establecido es:
\begin{enumerate}
 \item V8 Engine
 \item Node.js
 \item QOOXDOO
 \item Jquery
 \item Kendo
\end{enumerate}

Se verá por herramienta:
\begin{itemize}
 \item Descripción%, diseño.
 \item Características%, plataformas.
 \item How to hello\_world

\end{itemize}


Finalmente se realizarán comparaciones en una tabla resumen\\
%la cual tendrá 
%
%    i. Características
%   ii. Plataformas 
%  iii. Sistemas Operativos
%   iv. Uso de Plugins
%    v.

Cabe destacar que estas herramientas corresponden a frameworks de desarrollo. Pero ¿qué es un framework?
Corresponden a múltiples  herramientas asociadas entre sí, cuya finalidad son, en el caso del desarrollo web:
\begin{enumerate}
 \item No tener que lidiar con las particularidades de cada navegador.
 \item Escribir menos código fuente y hacer cosas más espectaculares.[1]
\end{enumerate}



[1] http://www.desarrolloweb.com/articulos/introduccion-jquery-mobile.html (hora de consulta 29-11-12 14:53)


Dudas:
\\
Nota(29-11-12 12:41), considerando que hay algunas herramientas, como qooxdoo y kendo que están
pensadas para más de una plataforma y tipo de aplicación (no necesariamente webapp, sino que desktop
o móviles), preguntar si es necesario adentrarse en estas "versiones" o solo que darse con la parte web.
