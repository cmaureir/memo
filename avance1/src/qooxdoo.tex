%Este apartado está destinado a revisar las acaracteristicas asociadas a la herramienta QOOXDOO

Sirve para crear una GUI con javascript en aplicaciones web, en lugar de crearlas con html y css es orientado a objetos, 
a diferencia de node.js q es orientado a eventos, es por ello que utiliza clases, interfaces (similares a las de java) y mezclas [3]\\

Se viene gestando desde 2005[4]\\

Las aplicaciones Web se suelen desarrollar utilizando XHTML, CSS y otras tecnologías similares. Sin embargo, 
el código de la aplicación de eyeOS está programado en JavaScript y se utiliza una biblioteca de gráficos llamado 
Qooxdoo (“QX”, para abreviar). Qooxdoo se utiliza para crear interfaces de usuario similares a las de otros escritorios
mediante el uso de herramientas como Qt, GTK +, o el Swing, lo que permite desarrollar una GUI (interfaz gráfica 
de usuario) con JavaScript en lugar de con XHTML y CSS. El enfoque de Qooxdoo permite a los programadores crear una 
interfaz similar a la de un escritorio de un ordenador. Qooxdoo es un proyecto de software de código abierto independiente 
de eyeOS, disponible en: qooxdoo.org[2]\\

Según la web oficial[3], es posible crear aplicaciones:
\begin{itemize}
 \item web
 \item moviles
 \item de escritorio 
 \item servidores
\end{itemize}

Es posible realizar pruebas sin hacer instalaciones [5]\\

Al igual que Jquery, es posible bajar un archivo js con el framework. Sin embargo también e sposible descargar un sdk que permite en pocos
pasos, realizar un "Hola mundo"

\subsection{Plataformas}

En algunos manuales, como en [6], se habla de Windows , Cygwin, Mac y Linux.\\




[1] http://www.ecured.cu/index.php/Qooxdoo (hora de consulta 28-11-12 22:59)

[2] http://www.eyeos.com/es/tecnologia/punto-de-vista-tecnico (hora de consulta 28-11-12 23:08)

[3] http://qooxdoo.org/ (hora de consulta 28-11-12 23:14)

[4] http://www.versioncero.com/noticia/202/qooxdoo (hora de consulta 28-11-12 23:24)

[5] http://demo.qooxdoo.org/current/playground (hora de consulta 28-11-12 23:28)



\subsection{Hello World}
%Se deja hasta acá la investigación y se sigue con kendo (29-11-12 01:09)
En [6], es posible ver un manual con la información necesaria para realizar el primer botón con hola mundo.
%(nota: Hasta este momento, no he podido descargar el sdk debido a problemas de conexion a internet en mi casa)

dentro de la carpeta del sdk
\begin{verbatim}
 tool/bin/create-application.py --name=custom --out=.
 cd custom/
 ./generate.py 
\end{verbatim}

Con eso se crea un archivo html que al abrirlo en el navegador, contiene un boton con el hola mundo.



[6] http://manual.qooxdoo.org/1.2/pages/getting\_started/helloworld.html (hora de consulta 01-12-12 17:18 )