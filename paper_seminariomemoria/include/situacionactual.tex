\section{Aplicaciones Web}

Una aplicación Web corresponde a un software basado en internet, donde una gran población de usuarios realizan peticiones
remotas a través de un navegador web y esperan res-puestas de un servidor web [2]\footnote{Página 90}. Corresponde a una aplicación que 
se codifica en un lenguaje soportado por los navegadores web.
Es decir las aplicaciones web corresponden al modelo \textbf{cliente-servidor}. 

Una aplicación web se distingue por el uso de \textit{hipermedia}, que conjuga tanto la tecnología hipertextual, como 
la multimedia. Si la multimedia proporciona una gran riqueza en los tipos de datos, el hipertexto aporta una estructura 
que permite que los datos puedan presentarse y explorarse siguiendo distintas secuencias, de acuerdo a las necesidades y 
preferencias del usuario.
%sacado de http://www.hipertexto.info/documentos/hipermedia.htm

En la actualidad existe una gran variedad de aplicaciones Web, que van desde páginas sólo de carácter informativo
hasta aplicaciones complejas que ofrecen diversidad de servicios al usuario, ya sean interacción
con otros usuarios a través de juegos en linea, mensajería y \textit{streaming} entre otras.
%sacado de Propuestas metodológicas para el desarrollo de aplicaciones Web: una evaluación según la ingeniería de métodos

Más allá del tipo de aplicación con el que se esté trabajando, es de suma importancia contar con un proceso de desarrollo de software,
con el fin de estandarizar su proceso de creación. A continuación se exponen algunos de los procesos de desarrollo de software
tanto general, como orientados al desarrollo de aplicaciones web.
%\textbf{Parrafo que indique la necesidad de utilizar algun proceso de desarrrollo de software}

\subsection{Procesos de Desarrollo de Software de carácter general}

Desde el punto de vista de la ingeniería de software, es importante contar con los mecanismos
adecuados (métodos y tecnologías de desarrollo, hardware acorde a la aplicaciones a desarrollar, entre
otros) para que la creación de este tipo de aplicaciones satisfaga las necesidades tanto de los usuarios 
como de los clientes que contratan el desarrollo de este tipo de aplicaciones.
En cualquier caso, existen criterios universalmente aceptados acerca del desarrollo software. Por ejemplo, 
el modelo de proceso más adecuado para el desarrollo de software es un proceso iterativo e incremental, 
puesto que a diferencia de otros modelos de proceso, como el modelo en cascada, permite la 
obtención de diversas versiones del producto software antes de su entrega final, por ende permite su depuración 
y validación progresiva, lo que sin duda redundará en un software de mejor calidad. Además, con este tipo de proceso es 
posible añadir o modificar requisitos que no han sido detectados con anterioridad [12]\footnote{Página 2}.

Dentro de los procesos de desarrollo de software más importantes, son destacables Modelo de cascada y el Modelo iterativo incremental

Estas metodologías se conocen también como "tradicionales" e imponen una disciplina de trabajo 
sobre el proceso de desarrollo del software, con la meta de conseguir uno más eficiente y predecible.
Para ello, se hace un especial hincapié en la planificación total de todo el trabajo a realizar y
una vez que esta todo detallado, comienza el ciclo de desarrollo del producto software. Este
planteamiento está basado en el resto de disciplinas de ingeniería, a pesar de que el software
no pueda considerarse como la construcción de una obra clásica de ingeniería [12]\footnote{Página 2.}.

Una de sus principales deficiencias corresponde a que si existe un cambio, es posible que toda la planificación se 
venga abajo, es por ello que no son métodos adecuados cuando se trabaja en un entorno que pueda variar constantemente.

%sacado de procesos agiles para el desarrollo de aplicaciones web

\subsection{Procesos de Desarrollo de Aplicaciones Web}

Si bien, actualmente no existe una metodología universalmente aceptada que guíe en el proceso de desarrollo de 
aplicaciones Web, se les suele asociar a los procesos de desarrollo denominados "ágiles" o "adptativos". %REVISAR ESTO ULTIMO.

Sin embargo, en los últimos años ha surgido un conjunto de métodos para desarrollar aplicaciones Web, los cuales presentan en forma
explícita su modelo de proceso, es decir las actividades técnicas y gerenciales que son requeridas para el
desarrollo de la aplicación. 
Los métodos que se presentarán se han agrupado de acuerdo a su modelo de proceso y al contexto particular donde pueden ser 
aplicados. Es necesario aclarar que, ninguno de ellos guía al grupo de desarrollo en la construcción de aplicaciones Web para múltiples 
contextos, debido a su entorno multivariable. Entre los métodos más conocidos se encuentran [2]\footnote{Páginas 90-92}:

\subsubsection{Métodos ágiles para el desarrollo de aplicaciones}
Estos procesos aportan como novedad, nuevos métodos de trabajo que apuestan por equilibrar la relación \textbf{proceso-esfuerzo}. 
En otras palabras, ni se pierden en la excesiva burocracia de los métodos tradicionales, ni apuestan por una ausencia total de procesos. 

Los procesos ágiles son una buena elección cuando se trabaja con requisitos desconocidos o variables. De no existir
requisitos estables, la posibilidad de utilizar un proceso "tradicional" no es recomendable. En estas situaciones, un 
proceso adaptativo se considera mucho más efectivo que un proceso predictivo. Los métodos ágiles para el desarrollo de software 
se caracterizan por poseer iteraciones cortas, pruebas continuas y frecuente replanificación basada en la realidad actual.

Dentro de las metodologías basadas en este enfoque destacan: Extreme Programming (XP), Open Source y Dynamic Systems 
Development Method (DSDM).


\subsubsection{Métodos para el desarrollo de sistemas de información Web (SIW) }
Estos métodos se caracterizan por seguir una secuencia de fases y pasos requeridos para desarrollar SIW, y asegurar la 
calidad del mismo. Cubren todo el ciclo de desarrollo de un SIW, emplean técnicas de análisis, de  diseño orientado
a objetos y utilizan un enfoque iterativo. 
Son comparables al proceso de desarrollo iterativo incremental, pues integra los siguientes procesos [11]:
\begin{itemize}
 \item Análisis preliminar para sistemas de información Web
 \item Selección de lenguaje y desarrollo de la aplicación
 \item Implementación
 \item Mantenimiento
 \item Estándares y Documentación
\end{itemize}

Uno de los más conocidos corresponde a la Metodología para desarrollar Sistemas de Información Web (MIDAS).

\subsubsection{Métodos para el desarrollo de aplicaciones hipermedia}
La mayoría de estos métodos sólo cubren parcialmente el ciclo de desarrollo de las aplicaciones hipermedia, dando
especial importancia al diseño. Por lo general utilizan dos técnicas en cualquier diseño de aplicaciones hipermedia:
Modelo Entidad-Relación y técnicas de Orientación a Objetos. 

Los más difundidos son: Aplicando modelos de proceso
de software al desarrollo de aplicaciones hipermedia, The Object-Oriented Hypermedia Design Model (OOHDM)
Relationship Management Methodology (RMM), Hypermedia Flexible Process Modeling (HFPM) y el Enfoque de Ingeniería 
de Lowe-Hall’s.

\subsubsection{Métodos para el desarrollo de sitios Web de aprendizaje (e-learning)}
El objetivo de estos métodos es ayudar a los diseñadores de los cursos y profesores a desarrollar sitios Web de aprendizaje
entendibles, por lo tanto se incorpora gran variedad de componentes organizacionales, administrativos, didacticos y 
tecnológicos, pues estos métodos consideran el elemento humano (alumnos, profesores, ayudantes, administradores), 
los recursos de aprendizaje basados en Web, otros recursos de aprendizaje (textos o guías) y la infraestructura tecnológica 
necesaria para desarrollar el proceso de aprendizaje. Cabe destacar que hacen énfasis en la reutilización de componentes para reducir el 
tiempo y costo de desarrollo. 

Los más conocidos son: Desarrollo de sitios Web instruccionales – Un Enfoque de Ingeniería de Software, Simple Web Method (SWM) 
y el Modelling Web-Based Instructional Systems.


\subsubsection{Métodos para el desarrollo de aplicaciones de comercio electrónico: (e-commerce)}
Estos métodos están basados en el reciclado e integración de componentes de software con el fin de lograr funcionalidades 
empresariales para aplicaciones de e-commerce (negociación, mediación; “workflow” interempresarial y notificaciones de
eventos). 

Uno de los más difundidos es el Marco de Referencia Basado en componentes para e-commerce.


