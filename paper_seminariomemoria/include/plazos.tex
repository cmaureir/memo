\section{Plan de trabajo, Tareas y Plazos}

Para definir los plazos necesarios de las diversas tareas a realizar, se usará como unidad de tiempo las semanas,
considerando el comienzo de la semana $0$ el momento de la inscripción del tema de memoria.

Aclarado esto, es posible definir una metodología de trabajo en base a grandes tareas, las cuales son:

\begin{enumerate}
 \item Investigación de las tecnologías, tanto existentes como emergentes para el desarrollo de aplicaciones web.
 \item Investigación y evaluación acerca de cómo éstas pueden ayudar a desarrollar mejores aplicaciones web y 
	mejorar los procesos de desarrollo de software.
 \item Definición de criterios que permitan realizar una clasificación general y una selección de aquellas 
	tecnologías específicas a investigar.
 \item Evaluación en mayor nivel de detalle de aquellas tecnologías seleccionadas.
 \item Definición de una aplicación de prueba y desarrollo de prototipo, aplicando las tecnologías seleccionadas, que permita entender, 
	integrar y evaluar estas tecnologías. 
 \item Diseño y desarrollo del prototipo.
 \item Evaluación global y obtención de conclusiones.
 \item Unión, recopilación de documentación para redactar informe final.
\end{enumerate}

\begin{enumerate}
 \item 3 semanas
 \item 1 semanas
 \item 2 semanas
 \item 5 semanas
 \item 5 semanas
 \item 7 semanas
 \item 3 semanas
\end{enumerate}

Lo cual suma alrededor de 26 semanas. Es necesario recalcar que esta estimación es solo preliminar, por tanto está sujeta
a posibles cambios. Además, es posible que se puedan paralelizar ciertas tareas para agilizar el proceso.


\subsection{Recursos a utilizar}

Para la realización de este proyecto, se utilizará:
\begin{itemize}
 \item Computador o máquina virtual con sistema operativo Unix y Windows; para probar las herramientas.
 \item Acceso a Internet, para descargar las herramientas y acceder a la documentación necesaria.
\end{itemize}

Cabe destacar que estos recursos son de carácter preliminar y estan sujetos a cambios.
